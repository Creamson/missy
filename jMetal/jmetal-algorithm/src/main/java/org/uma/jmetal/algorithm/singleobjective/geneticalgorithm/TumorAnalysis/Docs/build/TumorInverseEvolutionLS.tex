\documentclass[]{article}
\usepackage{lmodern}
\usepackage{amssymb,amsmath}
\usepackage{ifxetex,ifluatex}
\usepackage{fixltx2e} % provides \textsubscript
\ifnum 0\ifxetex 1\fi\ifluatex 1\fi=0 % if pdftex
  \usepackage[T1]{fontenc}
  \usepackage[utf8]{inputenc}
\else % if luatex or xelatex
  \ifxetex
    \usepackage{mathspec}
  \else
    \usepackage{fontspec}
  \fi
  \defaultfontfeatures{Ligatures=TeX,Scale=MatchLowercase}
\fi
% use upquote if available, for straight quotes in verbatim environments
\IfFileExists{upquote.sty}{\usepackage{upquote}}{}
% use microtype if available
\IfFileExists{microtype.sty}{%
\usepackage{microtype}
\UseMicrotypeSet[protrusion]{basicmath} % disable protrusion for tt fonts
}{}
\usepackage[margin=6em]{geometry}
\usepackage{hyperref}
\hypersetup{unicode=true,
            pdftitle={Model wzrostu guza jako problem odwrotny},
            pdfborder={0 0 0},
            breaklinks=true}
\urlstyle{same}  % don't use monospace font for urls
\usepackage{color}
\usepackage{fancyvrb}
\newcommand{\VerbBar}{|}
\newcommand{\VERB}{\Verb[commandchars=\\\{\}]}
\DefineVerbatimEnvironment{Highlighting}{Verbatim}{commandchars=\\\{\}}
% Add ',fontsize=\small' for more characters per line
\newenvironment{Shaded}{}{}
\newcommand{\KeywordTok}[1]{\textcolor[rgb]{0.00,0.44,0.13}{\textbf{{#1}}}}
\newcommand{\DataTypeTok}[1]{\textcolor[rgb]{0.56,0.13,0.00}{{#1}}}
\newcommand{\DecValTok}[1]{\textcolor[rgb]{0.25,0.63,0.44}{{#1}}}
\newcommand{\BaseNTok}[1]{\textcolor[rgb]{0.25,0.63,0.44}{{#1}}}
\newcommand{\FloatTok}[1]{\textcolor[rgb]{0.25,0.63,0.44}{{#1}}}
\newcommand{\ConstantTok}[1]{\textcolor[rgb]{0.53,0.00,0.00}{{#1}}}
\newcommand{\CharTok}[1]{\textcolor[rgb]{0.25,0.44,0.63}{{#1}}}
\newcommand{\SpecialCharTok}[1]{\textcolor[rgb]{0.25,0.44,0.63}{{#1}}}
\newcommand{\StringTok}[1]{\textcolor[rgb]{0.25,0.44,0.63}{{#1}}}
\newcommand{\VerbatimStringTok}[1]{\textcolor[rgb]{0.25,0.44,0.63}{{#1}}}
\newcommand{\SpecialStringTok}[1]{\textcolor[rgb]{0.73,0.40,0.53}{{#1}}}
\newcommand{\ImportTok}[1]{{#1}}
\newcommand{\CommentTok}[1]{\textcolor[rgb]{0.38,0.63,0.69}{\textit{{#1}}}}
\newcommand{\DocumentationTok}[1]{\textcolor[rgb]{0.73,0.13,0.13}{\textit{{#1}}}}
\newcommand{\AnnotationTok}[1]{\textcolor[rgb]{0.38,0.63,0.69}{\textbf{\textit{{#1}}}}}
\newcommand{\CommentVarTok}[1]{\textcolor[rgb]{0.38,0.63,0.69}{\textbf{\textit{{#1}}}}}
\newcommand{\OtherTok}[1]{\textcolor[rgb]{0.00,0.44,0.13}{{#1}}}
\newcommand{\FunctionTok}[1]{\textcolor[rgb]{0.02,0.16,0.49}{{#1}}}
\newcommand{\VariableTok}[1]{\textcolor[rgb]{0.10,0.09,0.49}{{#1}}}
\newcommand{\ControlFlowTok}[1]{\textcolor[rgb]{0.00,0.44,0.13}{\textbf{{#1}}}}
\newcommand{\OperatorTok}[1]{\textcolor[rgb]{0.40,0.40,0.40}{{#1}}}
\newcommand{\BuiltInTok}[1]{{#1}}
\newcommand{\ExtensionTok}[1]{{#1}}
\newcommand{\PreprocessorTok}[1]{\textcolor[rgb]{0.74,0.48,0.00}{{#1}}}
\newcommand{\AttributeTok}[1]{\textcolor[rgb]{0.49,0.56,0.16}{{#1}}}
\newcommand{\RegionMarkerTok}[1]{{#1}}
\newcommand{\InformationTok}[1]{\textcolor[rgb]{0.38,0.63,0.69}{\textbf{\textit{{#1}}}}}
\newcommand{\WarningTok}[1]{\textcolor[rgb]{0.38,0.63,0.69}{\textbf{\textit{{#1}}}}}
\newcommand{\AlertTok}[1]{\textcolor[rgb]{1.00,0.00,0.00}{\textbf{{#1}}}}
\newcommand{\ErrorTok}[1]{\textcolor[rgb]{1.00,0.00,0.00}{\textbf{{#1}}}}
\newcommand{\NormalTok}[1]{{#1}}
\usepackage{longtable,booktabs}
\usepackage{graphicx,grffile}
\makeatletter
\def\maxwidth{\ifdim\Gin@nat@width>\linewidth\linewidth\else\Gin@nat@width\fi}
\def\maxheight{\ifdim\Gin@nat@height>\textheight\textheight\else\Gin@nat@height\fi}
\makeatother
% Scale images if necessary, so that they will not overflow the page
% margins by default, and it is still possible to overwrite the defaults
% using explicit options in \includegraphics[width, height, ...]{}
\setkeys{Gin}{width=\maxwidth,height=\maxheight,keepaspectratio}
\IfFileExists{parskip.sty}{%
\usepackage{parskip}
}{% else
\setlength{\parindent}{0pt}
\setlength{\parskip}{6pt plus 2pt minus 1pt}
}
\setlength{\emergencystretch}{3em}  % prevent overfull lines
\providecommand{\tightlist}{%
  \setlength{\itemsep}{0pt}\setlength{\parskip}{0pt}}
\setcounter{secnumdepth}{5}
% Redefines (sub)paragraphs to behave more like sections
\ifx\paragraph\undefined\else
\let\oldparagraph\paragraph
\renewcommand{\paragraph}[1]{\oldparagraph{#1}\mbox{}}
\fi
\ifx\subparagraph\undefined\else
\let\oldsubparagraph\subparagraph
\renewcommand{\subparagraph}[1]{\oldsubparagraph{#1}\mbox{}}
\fi
\usepackage{mathrsfs}
\usepackage{amssymb}
\usepackage{empheq}
\usepackage{braket}
\usepackage{empheq}
\usepackage{graphicx}
\usepackage{float}
\usepackage{color}
\usepackage{listings}
\usepackage{mathtools}
\setcounter{secnumdepth}{4}
\setcounter{tocdepth}{4}

\title{Model wzrostu guza jako problem odwrotny}
\author{Leszek Siwik}
\date{2018-02-20}

\begin{document}
\maketitle
\begin{abstract}
Dokument stanowi (wstepny) raport z prac nad zdefiniowaniem problemu
wzrostu Tumora jako problemu odwrotnego wraz z próbą jego rozwiazywania
z wykorzystaniem ewolucji.
\end{abstract}

{
\setcounter{tocdepth}{3}
\tableofcontents
}
\newpage

\section{Background i punkt startowy}\label{background-i-punkt-startowy}

\label{sec:start} W listopadzie/grudniu 2017 przeprowadzono probe
analizy wrazliwosci modelu/solwera wykorzystywanego do symulacji wzrostu
Tumora w zależnosci od przyjetych parametrow.

Po drobnych doustaleniach jako referencyjne przyjeto nastepujace
wywoolanie/zestaw parametrow solwera:
\texttt{./tumor\ 2\ 80\ 10000\ 0.1\ 1000\ 0.5\ 10\ 2\ 10\ 100\ 0.001\ 0.3\ 0.625\ 0.3205\ 0.0064\ 0.0064\ 0.0000555\ 0.01\ 0.0000555\ 0.01\ 0.4\ 0.5\ 0.05\ 0.3\ 0.01333\ 10\ 0.003\ 2\ 5\ 25\ 24\ 0.003\ 0.4}

Zas dla poszczegolnych parametrow przyjeto w analizie nastepujacy zakres
ich zmiennosci:

\begin{itemize}
\tightlist
\item
  \(p_1=2\) -- stopien splajnów - nie zmieniany
\item
  \(p_2=80\) -- ilosc elementów nie zmieniany
\item
  \(p_3=10000\) -- liczba kroków czasowych nie zmieniany
\item
  \(p_4=0.1\) -- wielkosc kroku czasowego nie zmieniany
\item
  \(p_5=1000\) -- co ile krokow zapisywac wynik nie zmieniany
\item
  \(p_6=\tau_B = 0.5 \pm 10\%\)
\item
  \(p_7=o^{prol} = 10 \pm 10\%\)
\item
  \(p_8=o^{death} = 2 \pm 10\%\)
\item
  \(p_9=T^{prol} = 10 \pm 10\%\)
\item
  \(p_{10}=T^{death} = 100 \pm 10\%\)
\item
  \(p_{11}=P_b = 0.001 \pm 10\%\)
\item
  \(p_{12}=r_b\) -- zmieniany w zakresie \((0.00001,3/10]\)
\item
  \(p_{13}=\beta_M = 0.625 \pm 10\%\)
\item
  \(p_{14}=\gamma_A = 0.3205 \pm 10\%\)
\item
  \(p_{15}=\chi_{oA} = 0.0064 \pm 10\%\)
\item
  \(p_{16}=\gamma_{oA} = 0.0064 \pm 10\%\)
\item
  \(p_{17}=\chi_c = 0.0000555 \pm 10\%\)
\item
  \(p_{18}=\gamma_c = 0.01 \pm 10\%\)
\item
  \(p_{19}=\alpha_0 = 0.0000555 \pm \textbf{50\%}\)
\item
  \(p_{20}=\gamma_T = 0.01 \pm \textbf{50\%}\)
\item
  \(p_{21}=\alpha_1 = 0.4 \pm \textbf{50\%}\)
\end{itemize}

Podczas obliczen kazdy z wyzej wymienionych zakresow zmiennosci dzielono
na 20 rownych kawalkow i startujac od wywolania referencyjnego stopniowo
podstawiano kolejna z 21 wartosci parametru w zdefiniowanym zakresie
zmiennosci.

A zatem kolejne wywolania solwera dla parametru \(p_6\) definiowane byly
jako:

\begin{itemize}
\tightlist
\item
  ./tumor 2 80 10000 0.1 1000 \textbf{0.45} 10 2 10 100 0.001 0.3 0.625
  0.3205 0.0064 0.0064 0.0000555 0.01 0.0000555 0.01 0.4 0.5 0.05 0.3
  0.01333 10 0.003 2 5 25 24 0.003 0.4
\item
  ./tumor 2 80 10000 0.1 1000 \textbf{0.455} 10 2 10 100 0.001 0.3 0.625
  0.3205 0.0064 0.0064 0.0000555 0.01 0.0000555 0.01 0.4 0.5 0.05 0.3
  0.01333 10 0.003 2 5 25 24 0.003 0.4
\item
  \ldots{}
\item
  ./tumor 2 80 10000 0.1 1000 \textbf{0.55} 10 2 10 100 0.001 0.3 0.625
  0.3205 0.0064 0.0064 0.0000555 0.01 0.0000555 0.01 0.4 0.5 0.05 0.3
  0.01333 10 0.003 2 5 25 24 0.003 0.4
\end{itemize}

Nastepnie powtarzano procedure dla kolejnego parametru powracajac z
wartoscia wlasnie analizowanego parametru do wartosci referencyjnej
(\(0.5\) w powyzszym przypadku).

Uzyskane wyniki, oraz ich wstepna ``obróbka'' zebrana zostala w arkuszu
dostepnym
\textbf{\href{http://home.agh.edu.pl/~siwik/tumor/work.ods}{Tutaj}}.

Na pierwszy rzut oka, w uzyskanych wynikach w mojej opini nie widac
specjalnie sillnych zaleznosci, niemniej daje sie zidentyfikowac
parametry ktorych zmiana wartosci zdaje sie miec istotny ``kierunkowy''
wplyw na zachowanie modelu i wyniki uzyskiwane z implementujacego go
solwera.

W szczegolności jako parametry majace obserwowalny wplyw na wielkosc
tumora wyselekcjonowano parametry \(p_7=o^{prol}\), \(p_8=o^{death}\),
\(p_9=T^{prol}\), \(p_{10}=T^{death}\) a zidentyfikowana zaleznosc
pomiedzy wartosciami tych parametrow (w przyjetym zakresie zmiennosci) a
objetoscia guza zobrazowano na ponizszym wykresie.

\begin{figure}[htbp]
\centering
\includegraphics[width=0.80000\textwidth]{figures/1.png}
\caption{Zwiazek pomiedzy wartosciami parametrow \(p_7\), \(p_8\),
\(p_9\) i \(p_{10}\) a objetoscia guza}
\end{figure}

\section{Definicja problemu}\label{definicja-problemu}

Jako wstepny problem weryfikujacy mozliwosc dalszego researchu bazując
na obserwacjach i wynikach skomentowanych w poprzednim paragrafie
zdefiniowano problem odwrotny polegajacy na poszukiwaniu takich wartosci
parametrow \(p_7\), \(p_8\), \(p_9\) i \(p_{10}\) aby minimalizowac
odchylke uzyskiwanych wartosci objetosci guza od objetosci guza
uzyskanej dla symulacji referencyjnej.

Ta referencyjna wartość objetosci guza uzyskana podczas pierwszej fazy
eksperymentow to 402000. A zatem funkcja fitnes definiowana zostala
jako:

\begin{equation}
abs(402000-tumor_{volume})
 \end{equation}

Wartość ta byla minimalizowana podczas ewolucji.

\section{Użyty algorytm}\label{uux17cyty-algorytm}

\label{sec:algorytm}

\subsection{Algorytm}\label{algorytm}

Użyty w eksperymentach algorytm to klasyczny algorytm ewolucyjny, spięty
z solwerem symulującym rozrost guza działający zgodnie z przepisem jak
poniżej:

\begin{Shaded}
\begin{Highlighting}[]
\KeywordTok{public} \DataTypeTok{void} \FunctionTok{run}\NormalTok{() \{}
    \NormalTok{List<S> offspringPopulation;}
    \NormalTok{List<S> matingPopulation;}

    \NormalTok{population = }\FunctionTok{createInitialPopulation}\NormalTok{();}
    \NormalTok{population = }\FunctionTok{evaluatePopulation}\NormalTok{(population);}
    \FunctionTok{initProgress}\NormalTok{();}
    \KeywordTok{while} \NormalTok{(!}\FunctionTok{isStoppingConditionReached}\NormalTok{()) \{}
      \NormalTok{matingPopulation = }\FunctionTok{selection}\NormalTok{(population);}
      \NormalTok{offspringPopulation = }\FunctionTok{reproduction}\NormalTok{(matingPopulation);}
      \NormalTok{offspringPopulation = }\FunctionTok{evaluatePopulation}\NormalTok{(offspringPopulation);}
      \NormalTok{population = }\FunctionTok{replacement}\NormalTok{(population, offspringPopulation);}
      \FunctionTok{updateProgress}\NormalTok{();}
    \NormalTok{\}}
\NormalTok{\}}
\end{Highlighting}
\end{Shaded}

\subsection{Operatory wariacyjne}\label{operatory-wariacyjne}

Użyty algorytm wykorzystuje

\begin{itemize}
\item
  \textbf{Simulated Binary Crossover (SBXCrossover)}
  \footnote{Szczegoly np. tutaj: \url{https://pdfs.semanticscholar.org/b8ee/6b68520ae0291075cb1408046a7dff9dd9ad.pdf}}
  jako operator krzyzowania

\begin{Shaded}
\begin{Highlighting}[]
\NormalTok{crossoverOperator = }\KeywordTok{new} \FunctionTok{SBXCrossover}\NormalTok{(}\FloatTok{0.9}\NormalTok{,}\DecValTok{20}\NormalTok{);  }
\end{Highlighting}
\end{Shaded}
\item
  \textbf{mutacje wielomianowa} jako operator mutacji z nastepujacymi
  parametrami:

\begin{Shaded}
\begin{Highlighting}[]
\NormalTok{mutationOperator = }\KeywordTok{new} \FunctionTok{PolynomialMutation}\NormalTok{(}\FloatTok{1.0} \NormalTok{/ problem.}\FunctionTok{getNumberOfVariables}\NormalTok{(),}\DecValTok{20}\NormalTok{);}
\end{Highlighting}
\end{Shaded}
\item
  oraz \textbf{selekcje turniejowa} jako operator selekcji.
\end{itemize}

\subsection{Zrównoleglenie
algorytmu/solwera}\label{zruxf3wnoleglenie-algorytmusolwera}

Z uwagi na czas wykonania solwera na etapie ewaluacji osobników
zrównoleglono wywołania solwera.

Tzn. sam algorytm ewolucyjny nie był zrównoleglany w tym sensie, że nie
ma tam wielu przetwarzanych jednocześnie (sub)populacji etc. Mamy tam
``klasyczną'', pojedynczą globalnie przetwarzaną populację.

Natomiast zastosowano zrównoleglenie na etapie ewaluacji osobników. Tzn
dla każdego osbnika, dla którego potrzebna jest jego ewaluacja,
uruchamiany jest osobny proces/watek w którym wykonywany jest kod
solwera.

Zrownoleglewnie realizowane jest poprzez wywolania mpi:

\begin{Shaded}
\begin{Highlighting}[]
            \NormalTok{command = }\StringTok{"mpirun -np 1 -host node13:1 ../TumorGenetyka/tumor 2 80 9900 0.1 1000 0.5 "}
                    \NormalTok{+ p7 + }\StringTok{" "} \NormalTok{+ p8 + }\StringTok{" "} \NormalTok{+ p9 + }\StringTok{" "} \NormalTok{+ p10}
                    \NormalTok{+ }\StringTok{" 0.001 0.3 0.625 0.3205"}
                    \NormalTok{+ }\StringTok{" 0.0064 0.0064 0.0000555 0.01 0.0000555 0.01 0.4 0.5 0.05 0.3 0.01333"}
                    \NormalTok{+ }\StringTok{" 10 0.003 2 5 25 24 0.003 0.4"}\NormalTok{;}

            \NormalTok{p = Runtime.}\FunctionTok{getRuntime}\NormalTok{().}\FunctionTok{exec}\NormalTok{(}
                    \NormalTok{command);}
\end{Highlighting}
\end{Shaded}

\section{Wstępny eksperyment}\label{wstux119pny-eksperyment}

\subsection{Parametry algorytmu}\label{parametry-algorytmu}

Poniewaz natknieto sie na problemy z czasem wykonania pojedynczej
instancji solwera, w pierwszym opisywanym eksperymencie ograniczono
obliczenia do 100 ewaluacji (100 uruchomien solwera).

Rozmiar populacji algorytmu przyjęto na 20 (a zatem na etapie ewaluacji
uruchamiano max 20 procesów mpi wykonujących kod solwera.)

\subsection{Parametry modelu}\label{parametry-modelu}

Zakres zmienności parametrów (ich dopuszczalne granice) zdefiniowane
zostaly tak samo jak to miało miejsce przy analizie wrażliwosci modelu,
tzn:

\begin{itemize}
\tightlist
\item
  \(p_7=o^{prol} \in [9.0,11.0]\) (war ref. 10.0)
\item
  \(p_8=o^{death} \in [1.8,2.2]\) (war ref. 2.0)
\item
  \(p_9=T^{prol} \in [9.0,11.0]\) (war ref. 10.0)
\item
  \(p_{10}=T^{death} \in [90.0,110.0]\) (war ref. 100.0)
\end{itemize}

\subsection{Uzyskane wyniki}\label{uzyskane-wyniki}

\includegraphics[width=0.50000\textwidth]{figures/Tv1.png}
\includegraphics[width=0.50000\textwidth]{figures/p71.png}
\includegraphics[width=0.50000\textwidth]{figures/p81.png}
\includegraphics[width=0.50000\textwidth]{figures/p91.png}
\includegraphics[width=0.50000\textwidth]{figures/p101.png}

\begin{figure}[htbp]
\centering
\includegraphics{figures/t.png}
\caption{Results in first experiment\label{fig:first}}
\end{figure}

\subsection{Napotkane problemy}\label{napotkane-problemy}

Poniżej opisano problem na który nadziano się podczas wywoływania
solwera wraz z próbą wyjaśnienia jego przyczyn. Byc moze przyda sie w
analizie/poprawkach modelu/solvera ew. przyszłym użytkownikom
zaoszczędzi troche czasu :)

Klopot na jaki natrafiono to potezny wzrost czasu pojedynczego wykonania
solwera przy niektorych zestawach parametrow wejsciowych.

Jak bardzo to wydluazalo obliczenia: np w ``normalnych'' warunkach
pojedyncze wywolanie solwera trwalo ok 100s o tyle takie ``nienormalne''
wydluzalo sie np do 2500s albo w ogole sie nie konczylo bo powodowalo
tak duza alokacje pamieci ze bylo ubijane przez mpiruna.

M.in to bylo powodem wyczerpywania zasobow na wezle atari. Natrafiono na
to poniewaz poczatkowa konfiguracja alg ewolucyjnego zakladala wieksza
zmiennosc / szersze zakresy wartosci dopuszczalnych poszczegolnych
parametrow (zalozono np wstepnie zmiennosc \(p_7 \in [5,15]\),
\(p_8 \in [1,3]\), \(p_9 \in [5,15]\) i \(p_{10} \in [50,150]\))

Przykladowe wywolanie ktore bardzo dlugo sie liczylo to np:

\texttt{../TumorGenetyka/tumor\ 2\ 80\ 10000\ 0.1\ 1000\ 0.5\ 5.819810396051385\ 3.4400377040438777\ 11.112855386932164\ 125.35805356351595\ 0.001\ 0.3\ 0.625\ 0.3205\ 0.0064\ 0.0064\ 0.0000555\ 0.01\ 0.0000555\ 0.01\ 0.4\ 0.5\ 0.05\ 0.3\ 0.01333\ 10\ 0.003\ 2\ 5\ 25\ 24\ 0.003\ 0.4}

Po przyjrzeniu sie temu przypadkowi diagnoza jest taka ze to niefortunne
złożenie wartości parametrów. Parametry 7 - 10 to odpowiednio: 7 - ile
co najmniej tlenu potrzeba, żeby komórki się mnożyły 8 - ile co najwyżej
musi tlenu być, żeby żyły 9 - jak wolno się mnożą (większe - wolniej) 10
- jak wolno umierają

Uruchomienie które się nie doliczyło ma bardzo niską w porównaniu z
resztą wartość 7, czyli rak nie potrzebuje bardzo dużo tlenu żeby żyć,
więc będzie się rozrastał swobodnie. Jednocześnie ma bardzo niską watość
8, czyli musi mieć dość dużo tlenu żeby nie umierać - tam gdzie jest za
mało, jest tworzona substancja powodująca rozrost żył.

Różnica między tymi dwoma wartościami w tym uruchomieniu jest bardzo
mała (5.8 - 3.4 vs np. 14 - 1 w drugim), więc jest spora szansa że jak
rak będzie wołał o tlen to dostanie go od razu na tyle dużo, żeby móc
się rozrastać. Jednocześnie parametr 10 jest stosunkowo duży, więc nawet
jak nie dostanie tlenu od razu to nie będą komórki umierać bardzo
szybko.

Czyli podsumowując:

\begin{itemize}
\tightlist
\item
  rak będzie rosnąć jak dostanie nawet niewiele tlenu - rozejdzie się
  łatwo na większość dziedziny
\item
  jest ``wymagający'', będzie krzyczał o tlen (= rozrost żył) dopóki nie
  będzie miał prawie tyle żeby móc rosnąć, więc spowoduje gęste żyły
\item
  nie umiera szybko bez tlenu, więc może poczekać aż te żyły się utworzą
\end{itemize}

Teraz jakie z tego praktyczne wnioski\ldots{} trudno powiedzieć coś o
konkretnych zakresach parametrów. Być może należałoby je nieco zawęzić
(7 i 8 raczej będą najważniejsze), być może jakieś ograniczenia zakresu
na podstawie wartości jednego z nich (np. żeby różnica między 7 a 8 nie
była za duża) acz to chyba ciężko byłoby zrobić od strony algorytmu
genetycznego. Ale może to nie będzie potrzebne jak się zrobi to co jest
opisane dalej.

W przypadkach które się doliczyły też dość sporo się tych naczyń tworzy,
dużo więcej niż się spodziewałem. Jedna rzecz jaką na szybko można
zmienić która na pewno przyspieszy znacznie symulację w tej fazie z
gęstymi naczyniami to usunięcie łączenia naczyń które rosną tak, że
przez siebie przechodzą, bo to jest zaimplementowane z myślą o małych
ilościach naczyń i poszukuje bliskich węzłów brutalnie kwadratowo. Nie
powinno to raczej mieć specjalnie dużego wpływu na wyniki (tak mi się
wydaje).

Żeby to zrobić wystarczy zakomentować linijki 187-192 w
tumor/vasculature/vasculature.hpp:

\begin{verbatim}
                auto neighbor = find_neighbor(new_tip, tip, cfg.segment_length);
                if (neighbor != nullptr) {
                    connect(neighbor, new_tip);
                    removed = true;
                    it = sprouts.erase(it);
                }
\end{verbatim}

Co zostalo zrobione i wyniki z tego pierwsego eksperymentu mialy to w
solwerze zakomentowane.

Zeby zobrazować problem (być moze naprowadzić na rozwiazanie) ponizej
czasy wykonania solwera, ilosc segmentow i wezlow zyl, objetosc/masa
guza dla:

\begin{itemize}
\tightlist
\item
  przypadku normalnego (brak nadmiernego rozrostu zyl)
  \texttt{../TumorGenetyka/tumor\ 2\ 80\ 10000\ 0.1\ 1000\ 0.5\ 14.449578938363844\ 1.3853073609726758\ 7.860871658614398\ 63.030234570317525\ 0.001\ 0.3\ 0.625\ 0.3205\ 0.0064\ 0.0064\ 0.0000555\ 0.01\ 0.0000555\ 0.01\ 0.4\ 0.5\ 0.05\ 0.3\ 0.01333\ 10\ 0.003\ 2\ 5\ 25\ 24\ 0.003\ 0.4}
\item
  przypadku pesymistycznego (nadmierny rozrost)
  \texttt{../TumorGenetyka/tumor\ 2\ 80\ 10000\ 0.1\ 1000\ 0.5\ 5.819810396051385\ 3.4400377040438777\ 11.112855386932164\ 125.35805356351595\ 0.001\ 0.3\ 0.625\ 0.3205\ 0.0064\ 0.0064\ 0.0000555\ 0.01\ 0.0000555\ 0.01\ 0.4\ 0.5\ 0.05\ 0.3\ 0.01333\ 10\ 0.003\ 2\ 5\ 25\ 24\ 0.003\ 0.4}
\end{itemize}

\includegraphics[width=0.50000\textwidth]{figures/SolvTime1.png}
\includegraphics[width=0.50000\textwidth]{figures/VascNodes1.png}
\includegraphics[width=0.50000\textwidth]{figures/VascSeg1.png}
\includegraphics[width=0.50000\textwidth]{figures/Tv2.png}

\begin{figure}[htbp]
\centering
\includegraphics{figures/t.png}
\caption{Czas obliczen, siatka zyl, masa guza w ``normalnym'' i
``pesymistycznym'' przebiegu symulacji\label{fig:first}}
\end{figure}

\section{\texorpdfstring{Ewolucyjne poszukiwanie wartości referencyjnych
wybranych parametrow startując z `przypadkowego' zakresu ich
wartosci}{Ewolucyjne poszukiwanie wartości referencyjnych wybranych parametrow startując z przypadkowego zakresu ich wartosci}}\label{ewolucyjne-poszukiwanie-wartoux15bci-referencyjnych-wybranych-parametrow-startujux105c-z-przypadkowego-zakresu-ich-wartosci}

Po opisanym wcześniej wstępnym zweryfikowaniu działania narzędzi -
przystapiono do przeprowadzenia eksperymentu polegającego w uproszczeniu
na zweryfikowaniu zdolności odnalezienia przez wczesniej opisany solwer
(hybryda alg. ewolucyjnego z solwerem modelujacym wzrost guza)\\
referencyjnych wartości wybranych parametrów startując z innego zakresu
ich wartości.

Podczas eksperymentu stopniowo zwiększamy `stopień trudności' tzn
zaburzeniu ulega stopniowo (w stosunku do zakresu referencyjnego)
rosnąca ilość parametrów (tych przyjętych na wstępnym etapie weryfikacji
jako istotne czyli odpowiednio: \(p_7=o^{prol}\), \(p_8=o^{death}\),
\(p_9=T^{prol}\) oraz \(p_{10}=T^{death}\)) czyli:

\begin{itemize}
\tightlist
\item
  zaburzeniu podlega 1 parameter (4 symulacje po jednej dla
  `indywidualnie' zaburzonych parametrów \(p_7\ldots p_{10}\)),
\item
  zaburzeniu podlegaja 2 parametry (6 symulacji dla roznych par
  zaburzonych parametrów)
\item
  zaburzeniu podlegaja 3 parametry (4 symulacje dla jednego parametru
  niezaburzonego)
\item
  zaburzeniu podlegaja 4 parametry (1 symulacja dla wszystkich
  zaburzonych )
\end{itemize}

\subsection{Seria 1 - zaburzanie pojedynczego
parametru}\label{seria-1---zaburzanie-pojedynczego-parametru}

Zgodnie z wcześniejszym opisem pierwsza seria eksperymenów polega na
pojedynczym zaburzaniu parametrow \(p_7\ldots p_{10}\) i weryfikacji
możliwości odnalezienia przez hybrydowy solwer wartości referencyjnych
tych parametrów Eksperymenty przeprowadzano w ten sposób, że na poziomie
alg ewolucyjnego zmieniano zakres dopuszczalnych wartości zaburzanego
parametru dla populacji początkowej , a następnie rozszerzano zakres
dopuszczalny tego parametru w taki sposób aby obejmowal on zakres
referencyjny oraz ten wynikający z zaburzenia.

Innymi słowy populacja początkowa zawierała osobniki z zaburzonymi
wartościami wybranego parametru (wybranych parametrów) natomiast
wartości tego parametru dla osbników tworzonych w trakcie działania alg.
ewolucyjnego generowane były z szerszego zakresu tj.
\(zakres_{referencyjny} \uplus zakres_{zaburzony} \uplus zakres_{dodatkowy}\)

Wszystkie pozostałe parametry zarówno alg ewolucyjnego jak i wywołań
symulatora wzrostu guza pozostawały niezmienione i były takie same jak
opisane w sekcjach \ref{sec:algorytm} oraz \ref{sec:start} zwiekszając
ilość ewaluacji ewolucji do 200.

Planowane do przyjęcia podczas eksperymentów zakresy zmienności
poszczególnych analizowanych parametrów:

\begin{longtable}[c]{@{}lccc@{}}
\toprule
\begin{minipage}[b]{0.13\columnwidth}\raggedright\strut
Parametr
\strut\end{minipage} &
\begin{minipage}[b]{0.26\columnwidth}\centering\strut
Zakres zmienności podczas wstępnej ewaluacji
\strut\end{minipage} &
\begin{minipage}[b]{0.24\columnwidth}\centering\strut
Zakres zmienności dla populacji początkowej
\strut\end{minipage} &
\begin{minipage}[b]{0.26\columnwidth}\centering\strut
Zakres zmienności w trakcie trwania ewolucji
\strut\end{minipage}\tabularnewline
\midrule
\endhead
\begin{minipage}[t]{0.13\columnwidth}\raggedright\strut
\(p_7\)
\strut\end{minipage} &
\begin{minipage}[t]{0.26\columnwidth}\centering\strut
\(p_7 = 10 \pm 10\%\)
\strut\end{minipage} &
\begin{minipage}[t]{0.24\columnwidth}\centering\strut
\(p_7 \in [11.0,15.0]\)
\strut\end{minipage} &
\begin{minipage}[t]{0.26\columnwidth}\centering\strut
\(p_7 \in [5.0,15.0]\)
\strut\end{minipage}\tabularnewline
\begin{minipage}[t]{0.13\columnwidth}\raggedright\strut
\(p_8\)
\strut\end{minipage} &
\begin{minipage}[t]{0.26\columnwidth}\centering\strut
\(p_8 = 2 \pm 10 \%\)
\strut\end{minipage} &
\begin{minipage}[t]{0.24\columnwidth}\centering\strut
\(p_{8} \in [2.1,3.0]\)
\strut\end{minipage} &
\begin{minipage}[t]{0.26\columnwidth}\centering\strut
\(p_8 \in [1.0,3.0]\)
\strut\end{minipage}\tabularnewline
\begin{minipage}[t]{0.13\columnwidth}\raggedright\strut
\(p_9\)
\strut\end{minipage} &
\begin{minipage}[t]{0.26\columnwidth}\centering\strut
\(p_9 = 10 \pm 10 \%\)
\strut\end{minipage} &
\begin{minipage}[t]{0.24\columnwidth}\centering\strut
\(p_9 \in [11.0,15.0]\)
\strut\end{minipage} &
\begin{minipage}[t]{0.26\columnwidth}\centering\strut
\(p_9 \in [5.0,15.0]\)
\strut\end{minipage}\tabularnewline
\begin{minipage}[t]{0.13\columnwidth}\raggedright\strut
\(p_{10}\)
\strut\end{minipage} &
\begin{minipage}[t]{0.26\columnwidth}\centering\strut
\(p_{10} = 100 \pm 10 \%\)
\strut\end{minipage} &
\begin{minipage}[t]{0.24\columnwidth}\centering\strut
\(p_{10} \in [110.0,150.0]\)
\strut\end{minipage} &
\begin{minipage}[t]{0.26\columnwidth}\centering\strut
\(p_{10} \in [50.0,150.0]\)
\strut\end{minipage}\tabularnewline
\bottomrule
\end{longtable}

\pagebreak
\pagebreak

\subsubsection{\texorpdfstring{Zaburzanie parametru
\(p_7=o^{prol}\)}{Zaburzanie parametru p\_7=o\^{}\{prol\}}}\label{zaburzanie-parametru-pux5f7oprol}

\paragraph{Parametry eksperymentu}\label{parametry-eksperymentu}

Dla przypomnienia parametr \(p_7=o^{prol}\) solwera mówi ile co najmniej
tlenu potrzeba, żeby komórki się mnożyły. Na wstępnym etapie jako
wartość referencyjną tego parametru przyjęto jako
\(p_7=o^{prol} \in [9.0,11.0]\) (z war ref. 10.0).

W pierwszym eksperymencie przyjęto zatem, iż początkowa seria wartości
tego parametru (wartości tego parametru dla populacji początkowej alg
ewolucyjnego) losowana będzie z przedziału \(p_7' \in [11.0,15.0]\)
Natomiast dla osobników generowanych już w trakcie działania alg.
ewolucyjnego jako dopuszczalny zakres wartości parametru \(p_7\)
przyjęto zakres \(p_7 \in [5.0,15.0]\)

\begin{longtable}[c]{@{}lccc@{}}
\toprule
\begin{minipage}[b]{0.13\columnwidth}\raggedright\strut
Parametr
\strut\end{minipage} &
\begin{minipage}[b]{0.26\columnwidth}\centering\strut
Zakres zmienności podczas wstępnej ewaluacji
\strut\end{minipage} &
\begin{minipage}[b]{0.24\columnwidth}\centering\strut
Zakres zmienności dla populacji początkowej
\strut\end{minipage} &
\begin{minipage}[b]{0.26\columnwidth}\centering\strut
Zakres zmienności w trakcie trwania ewolucji
\strut\end{minipage}\tabularnewline
\midrule
\endhead
\begin{minipage}[t]{0.13\columnwidth}\raggedright\strut
\(p_7\)
\strut\end{minipage} &
\begin{minipage}[t]{0.26\columnwidth}\centering\strut
\(p_7 = 10 \pm 10\%\)
\strut\end{minipage} &
\begin{minipage}[t]{0.24\columnwidth}\centering\strut
\(p_7 \in [11.0,15.0]\)
\strut\end{minipage} &
\begin{minipage}[t]{0.26\columnwidth}\centering\strut
\(p_7 \in [5.0,15.0]\)
\strut\end{minipage}\tabularnewline
\bottomrule
\end{longtable}

\paragraph{Otrzymane wyniki}\label{otrzymane-wyniki}

Ponizej odpowiednio: rozklad p7 w osobnikach na kolejnych etapach
działania alg ewolucyjnego, jego wartosci usrednione, maksymalne i
minimalne w populacji, oraz odchylka sredniej wartosci \(p_7\) w
populacji od wartosci referencyjnej w kolejnych iteracjach.

\begin{figure}[htbp]
\centering
\includegraphics[width=1.00000\textwidth]{figures/popp7.png}
\caption{Rozklad \(p_7\) w populacji}
\end{figure}

\begin{figure}[htbp]
\centering
\includegraphics[width=1.00000\textwidth]{figures/agrP7.png}
\caption{Wartosci srednie, maksymalne oraz minimalne p7 na kolejnych
etapach dzialania alg ewolucyjnego}
\end{figure}

\begin{figure}[htbp]
\centering
\includegraphics[width=1.00000\textwidth]{figures/Odchylkap7.png}
\caption{Odchylka sredniej wartosci p7 w populacji od wartosci
referencyjnej}
\end{figure}

\pagebreak

\subsubsection{\texorpdfstring{Zaburzanie parametru
\(p_8=o^{death}\)}{Zaburzanie parametru p\_8=o\^{}\{death\}}}\label{zaburzanie-parametru-pux5f8odeath}

\paragraph{Parametry eksperymentu}\label{parametry-eksperymentu-1}

Dla przypomnienia parametr \(p_8=o^{death}\) solwera mówi ile ile co
najwyżej musi być tlenu, żeby żyły. Na wstępnym etapie jako wartość
referencyjną tego parametru przyjęto jako
\(p_8=o^{death} \in [1.8,2.2]\) (war ref. 2.0).

W niniejszym eksperymencie przyjęto zatem, iż początkowa seria wartości
tego parametru (wartości tego parametru dla populacji początkowej alg
ewolucyjnego) losowana będzie z przedziału \(p_7' \in [2.1,3.0]\)
Natomiast dla osobników generowanych już w trakcie działania alg.
ewolucyjnego jako dopuszczalny zakres wartości parametru \(p_8\)
przyjęto zakres \(p_8 \in [1.0,3.0]\)

\begin{longtable}[c]{@{}lccc@{}}
\toprule
\begin{minipage}[b]{0.13\columnwidth}\raggedright\strut
Parametr
\strut\end{minipage} &
\begin{minipage}[b]{0.26\columnwidth}\centering\strut
Zakres zmienności podczas wstępnej ewaluacji
\strut\end{minipage} &
\begin{minipage}[b]{0.24\columnwidth}\centering\strut
Zakres zmienności dla populacji początkowej
\strut\end{minipage} &
\begin{minipage}[b]{0.26\columnwidth}\centering\strut
Zakres zmienności w trakcie trwania ewolucji
\strut\end{minipage}\tabularnewline
\midrule
\endhead
\begin{minipage}[t]{0.13\columnwidth}\raggedright\strut
\(p_8\)
\strut\end{minipage} &
\begin{minipage}[t]{0.26\columnwidth}\centering\strut
\(p_8 = 2 \pm 10 \%\)
\strut\end{minipage} &
\begin{minipage}[t]{0.24\columnwidth}\centering\strut
\(p_{8} \in [2.1,3.0]\)
\strut\end{minipage} &
\begin{minipage}[t]{0.26\columnwidth}\centering\strut
\(p_8 \in [1.0,3.0]\)
\strut\end{minipage}\tabularnewline
\bottomrule
\end{longtable}

\paragraph{Otrzymane wyniki}\label{otrzymane-wyniki-1}

Ponizej odpowiednio: rozklad \(p_8\) w osobnikach na kolejnych etapach
działania alg ewolucyjnego, jego wartosci usrednione, maksymalne i
minimalne w populacji, oraz odchylka sredniej wartosci \(p_7\) w
populacji od wartosci referencyjnej w kolejnych iteracjach.

\begin{figure}[htbp]
\centering
\includegraphics[width=1.00000\textwidth]{figures/popp8.png}
\caption{Rozklad \(p_8\) w populacji}
\end{figure}

\begin{figure}[htbp]
\centering
\includegraphics[width=1.00000\textwidth]{figures/agrP8.png}
\caption{Wartosci srednie, maksymalne oraz minimalne p8 na kolejnych
etapach dzialania alg ewolucyjnego}
\end{figure}

\begin{figure}[htbp]
\centering
\includegraphics[width=1.00000\textwidth]{figures/Odchylkap8.png}
\caption{Odchylka sredniej wartosci p8 w populacji od wartosci
referencyjnej}
\end{figure}

\paragraph{Napotkane problemy}\label{napotkane-problemy-1}

Przy eksperymentowaniu z zaburzonym p8 nadziano sie na klopociki
prawdopodobnie bedace pochodna/kontynuacja problemów opisywanych
wczesniej polegajacych na ``niestabilnym'' dzialaniu solwera przy
niektorych wartosciach/kombinacjach parametrow

Ponizej wywolania solwera ktore sie nie powiodly (moze przyda sie kiedys
do poprawek w solwerze)

Unable to run solver: java.lang.NumberFormatException: For input string:
``===================================================================================''
mpirun -np 1 -host node13:1 ../TumorGenetyka/tumor 2 80 9900 0.1 1000
0.5 9.454590847505983 2.9222689984674073 9.063008773943382
96.1524747998914 0.001 0.3 0.625 0.3205 0.0064 0.0064 0.0000555 0.01
0.0000555 0.01 0.4 0.5 0.05 0.3 0.01333 10 0.003 2 5 25 24 0.003 0.4
Unable to run solver: java.lang.NumberFormatException: For input string:
``===================================================================================''
mpirun -np 1 -host node13:1 ../TumorGenetyka/tumor 2 80 9900 0.1 1000
0.5 9.430679318110801 2.9613064430274214 9.001731455509889
94.84556414555985 0.001 0.3 0.625 0.3205 0.0064 0.0064 0.0000555 0.01
0.0000555 0.01 0.4 0.5 0.05 0.3 0.01333 10 0.003 2 5 25 24 0.003 0.4
Unable to run solver: java.lang.NumberFormatException: For input string:
``===================================================================================''
mpirun -np 1 -host node13:1 ../TumorGenetyka/tumor 2 80 9900 0.1 1000
0.5 9.312816143866291 2.8373159607698737 9.068744087595107
109.4082352110131 0.001 0.3 0.625 0.3205 0.0064 0.0064 0.0000555 0.01
0.0000555 0.01 0.4 0.5 0.05 0.3 0.01333 10 0.003 2 5 25 24 0.003 0.4
Unable to run solver: java.lang.NumberFormatException: For input string:
``===================================================================================''
mpirun -np 1 -host node13:1 ../TumorGenetyka/tumor 2 80 9900 0.1 1000
0.5 9.45412400070996 2.9512270566632774 9.134113928012875
96.1524747998914 0.001 0.3 0.625 0.3205 0.0064 0.0064 0.0000555 0.01
0.0000555 0.01 0.4 0.5 0.05 0.3 0.01333 10 0.003 2 5 25 24 0.003 0.4
Unable to run solver: java.lang.NumberFormatException: For input string:
``===================================================================================''
mpirun -np 1 -host node13:1 ../TumorGenetyka/tumor 2 80 9900 0.1 1000
0.5 9.506700280337986 2.6729585594183844 9.127711756586537
98.28336497095415 0.001 0.3 0.625 0.3205 0.0064 0.0064 0.0000555 0.01
0.0000555 0.01 0.4 0.5 0.05 0.3 0.01333 10 0.003 2 5 25 24 0.003 0.4
Unable to run solver: java.lang.NumberFormatException: For input string:
``===================================================================================''
mpirun -np 1 -host node13:1 ../TumorGenetyka/tumor 2 80 9900 0.1 1000
0.5 9.428205108285592 2.6603929941110316 9.0677395656585
97.18615629884131 0.001 0.3 0.625 0.3205 0.0064 0.0064 0.0000555 0.01
0.0000555 0.01 0.4 0.5 0.05 0.3 0.01333 10 0.003 2 5 25 24 0.003 0.4
Unable to run solver: java.lang.NumberFormatException: For input string:
``===================================================================================''
mpirun -np 1 -host node13:1 ../TumorGenetyka/tumor 2 80 9900 0.1 1000
0.5 9.477124985184965 2.7878387123218205 9.071758596198237
108.77922017134355 0.001 0.3 0.625 0.3205 0.0064 0.0064 0.0000555 0.01
0.0000555 0.01 0.4 0.5 0.05 0.3 0.01333 10 0.003 2 5 25 24 0.003 0.4
Read from err: terminate Read from err: called Read from err: after Read
from err: throwing Read from err: an Read from err: instance Read from
err: of Read from err: `std::bad\_alloc' Read from err: what(): Read
from err: std::bad\_alloc Unable to run solver:
java.lang.NumberFormatException: For input string:
``===================================================================================''
mpirun -np 1 -host node13:1 ../TumorGenetyka/tumor 2 80 9900 0.1 1000
0.5 9.45300605302644 2.991957323550271 9.06939941685454 96.1524747998914
0.001 0.3 0.625 0.3205 0.0064 0.0064 0.0000555 0.01 0.0000555 0.01 0.4
0.5 0.05 0.3 0.01333 10 0.003 2 5 25 24 0.003 0.4 Unable to run solver:
java.lang.NumberFormatException: For input string:
``===================================================================================''
mpirun -np 1 -host node13:1 ../TumorGenetyka/tumor 2 80 9900 0.1 1000
0.5 9.424280747231732 2.816132673178266 9.080270122482826
109.95693142895279 0.001 0.3 0.625 0.3205 0.0064 0.0064 0.0000555 0.01
0.0000555 0.01 0.4 0.5 0.05 0.3 0.01333 10 0.003 2 5 25 24 0.003 0.4
Unable to run solver: java.lang.NumberFormatException: For input string:
``===================================================================================''
mpirun -np 1 -host node13:1 ../TumorGenetyka/tumor 2 80 9900 0.1 1000
0.5 9.454330657752786 2.6073509233841365 9.065938202205565
109.70008853564136 0.001 0.3 0.625 0.3205 0.0064 0.0064 0.0000555 0.01
0.0000555 0.01 0.4 0.5 0.05 0.3 0.01333 10 0.003 2 5 25 24 0.003 0.4
Unable to run solver: java.lang.NumberFormatException: For input string:
``===================================================================================''
mpirun -np 1 -host node13:1 ../TumorGenetyka/tumor 2 80 9900 0.1 1000
0.5 9.729087704398712 2.9222689984674073 9.146183592522865
98.30290819602372 0.001 0.3 0.625 0.3205 0.0064 0.0064 0.0000555 0.01
0.0000555 0.01 0.4 0.5 0.05 0.3 0.01333 10 0.003 2 5 25 24 0.003 0.4

Unable to run solver: java.lang.NumberFormatException: For input string:
``===================================================================================''
mpirun -np 1 -host node13:1 ../TumorGenetyka/tumor 2 80 9900 0.1 1000
0.5 9.312825087735012 2.9227547222493135 9.24885580468986
90.5577187348602 0.001 0.3 0.625 0.3205 0.0064 0.0064 0.0000555 0.01
0.0000555 0.01 0.4 0.5 0.05 0.3 0.01333 10 0.003 2 5 25 24 0.003 0.4
Unable to run solver: java.lang.NumberFormatException: For input string:
``===================================================================================''
mpirun -np 1 -host node13:1 ../TumorGenetyka/tumor 2 80 9900 0.1 1000
0.5 9.45412400070996 2.9512270566632774 9.11652843068546
96.1524747998914 0.001 0.3 0.625 0.3205 0.0064 0.0064 0.0000555 0.01
0.0000555 0.01 0.4 0.5 0.05 0.3 0.01333 10 0.003 2 5 25 24 0.003 0.4
Unable to run solver: java.lang.NumberFormatException: For input string:
``===================================================================================''
mpirun -np 1 -host node13:1 ../TumorGenetyka/tumor 2 80 9900 0.1 1000
0.5 9.454330657752786 2.7220574749335147 9.065931652601307
109.70008853564136 0.001 0.3 0.625 0.3205 0.0064 0.0064 0.0000555 0.01
0.0000555 0.01 0.4 0.5 0.05 0.3 0.01333 10 0.003 2 5 25 24 0.003 0.4
Unable to run solver: java.lang.NumberFormatException: For input string:
``===================================================================================''
mpirun -np 1 -host node13:1 ../TumorGenetyka/tumor 2 80 9900 0.1 1000
0.5 9.455134787600183 2.922261983693214 9.063918179902238
109.55449860456025 0.001 0.3 0.625 0.3205 0.0064 0.0064 0.0000555 0.01
0.0000555 0.01 0.4 0.5 0.05 0.3 0.01333 10 0.003 2 5 25 24 0.003 0.4
Unable to run solver: java.lang.NumberFormatException: For input string:
``===================================================================================''
mpirun -np 1 -host node13:1 ../TumorGenetyka/tumor 2 80 9900 0.1 1000
0.5 9.006641892977663 2.8373206073973387 9.002174195946964
93.65986847646002 0.001 0.3 0.625 0.3205 0.0064 0.0064 0.0000555 0.01
0.0000555 0.01 0.4 0.5 0.05 0.3 0.01333 10 0.003 2 5 25 24 0.003 0.4
Unable to run solver: java.lang.NumberFormatException: For input string:
``===================================================================================''
mpirun -np 1 -host node13:1 ../TumorGenetyka/tumor 2 80 9900 0.1 1000
0.5 9.454160715570499 2.9956881333657854 9.001542159250945
94.68929923984707 0.001 0.3 0.625 0.3205 0.0064 0.0064 0.0000555 0.01
0.0000555 0.01 0.4 0.5 0.05 0.3 0.01333 10 0.003 2 5 25 24 0.003 0.4
Unable to run solver: java.lang.NumberFormatException: For input string:
``===================================================================================''
mpirun -np 1 -host node13:1 ../TumorGenetyka/tumor 2 80 9900 0.1 1000
0.5 9.401906819035798 2.996356833670716 9.254686609769006
90.49447838526919 0.001 0.3 0.625 0.3205 0.0064 0.0064 0.0000555 0.01
0.0000555 0.01 0.4 0.5 0.05 0.3 0.01333 10 0.003 2 5 25 24 0.003 0.4
Unable to run solver: java.lang.NumberFormatException: For input string:
``===================================================================================''
mpirun -np 1 -host node13:1 ../TumorGenetyka/tumor 2 80 9900 0.1 1000
0.5 9.453536622315424 2.9424569354815837 9.249968498181236
96.1524747998914 0.001 0.3 0.625 0.3205 0.0064 0.0064 0.0000555 0.01
0.0000555 0.01 0.4 0.5 0.05 0.3 0.01333 10 0.003 2 5 25 24 0.003 0.4
Unable to run solver: java.lang.NumberFormatException: For input string:
``===================================================================================''
mpirun -np 1 -host node13:1 ../TumorGenetyka/tumor 2 80 9900 0.1 1000
0.5 9.454590847505983 2.9222689984674073 9.25062940767948
96.92921554534925 0.001 0.3 0.625 0.3205 0.0064 0.0064 0.0000555 0.01
0.0000555 0.01 0.4 0.5 0.05 0.3 0.01333 10 0.003 2 5 25 24 0.003 0.4
Unable to run solver: java.lang.NumberFormatException: For input string:
``===================================================================================''
mpirun -np 1 -host node13:1 ../TumorGenetyka/tumor 2 80 9900 0.1 1000
0.5 9.422952034739607 2.923597621538141 9.005725604299714
96.1524747998914 0.001 0.3 0.625 0.3205 0.0064 0.0064 0.0000555 0.01
0.0000555 0.01 0.4 0.5 0.05 0.3 0.01333 10 0.003 2 5 25 24 0.003 0.4

\pagebreak
\pagebreak

\subsubsection{\texorpdfstring{Zaburzanie parametru
\(p_9=T^{prol}\)}{Zaburzanie parametru p\_9=T\^{}\{prol\}}}\label{zaburzanie-parametru-pux5f9tprol}

\paragraph{Parametry eksperymentu}\label{parametry-eksperymentu-2}

Dla przypomnienia parametr \(p_9=T^{prol}\) solwera mówi jak wolno
komórki się mnożą (większe - wolniej) Na wstępnym etapie jako wartość
referencyjną tego parametru przyjęto jako
\(p_9=T^{prol} \in [9.0,11.0]\) (war ref. 10.0)

W niniejszym eksperymencie przyjęto zatem, iż początkowa seria wartości
tego parametru (wartości tego parametru dla populacji początkowej alg
ewolucyjnego) losowana będzie z przedziału \(p_9' \in [11.0,15.0]\)
Natomiast dla osobników generowanych już w trakcie działania alg.
ewolucyjnego jako dopuszczalny zakres wartości parametru \(p_9\)
przyjęto zakres \(p_9 \in [5.0,15.0]\)

\paragraph{Otrzymane wyniki}\label{otrzymane-wyniki-2}

Eksperyment w trakcie realizacji\ldots{}.

\pagebreak
\pagebreak

\subsubsection{\texorpdfstring{Zaburzanie parametru
\(p_{10}=T^{death}\)}{Zaburzanie parametru p\_\{10\}=T\^{}\{death\}}}\label{zaburzanie-parametru-pux5f10tdeath}

\paragraph{Parametry eksperymentu}\label{parametry-eksperymentu-3}

Dla przypomnienia parametr \(p_{10}=T^{death}\) solwera mówi jak wolno
umierają komórki. Na wstępnym etapie jako wartość referencyjną tego
parametru przyjęto jako \(p_{10}=T^{death} \in [90.0,110.0]\) (war ref.
100.0)

W niniejszym eksperymencie przyjęto zatem, iż początkowa seria wartości
tego parametru (wartości tego parametru dla populacji początkowej alg
ewolucyjnego) losowana będzie z przedziału \(p_{10}' \in [110.0,150.0]\)
Natomiast dla osobników generowanych już w trakcie działania alg.
ewolucyjnego jako dopuszczalny zakres wartości parametru \(p_{10}\)
przyjęto zakres \(p_{10} \in [50.0,150.0]\)

\paragraph{Otrzymane wyniki}\label{otrzymane-wyniki-3}

Eksperyment w trakcie realizacji \ldots{}.

\textbf{Eksperymentu ostatecznie nie wykonanu}

\subsection{Kwestie do zweryfikowania}\label{kwestie-do-zweryfikowania}

\begin{itemize}
\tightlist
\item
  czy zaburzanie realizowane jak powyzej - ok?
\item
  czy pprzyjmowana sila zaburzenia / zakres zmian parametru w pop
  poczatkowej ok/wystarczajacy?
\item
  czy parametry ewolucji (ilosc ewaluacji / dlugosc trwania ok)?
\item
  czy zakres gromadzonych / wizualizowanych wyników wystarczajacy? (moze
  potrzebne cos jeszcze? czas obliczen? masa guza? \ldots{} )
\end{itemize}

\newpage

\section{Eksperyment z fixowaniem wartosci parametrow niepodlegajacych
ewolucji/poszukiwaniu}\label{eksperyment-z-fixowaniem-wartosci-parametrow-niepodlegajacych-ewolucjiposzukiwaniu}

\subsection{Sposób przeprowadzenia
eksperymentow}\label{sposuxf3b-przeprowadzenia-eksperymentow}

W ponizszej serii eksperymentow podobnie jak wczesniej stopniowo
zwiększamy `stopień trudności' tzn zaburzeniu ulega stopniowo (w
stosunku do zakresu referencyjnego) rosnąca ilość parametrów (tych
przyjętych na wstępnym etapie weryfikacji jako istotne czyli
odpowiednio: \(p_7=o^{prol}\), \(p_8=o^{death}\), \(p_9=T^{prol}\) oraz
\(p_{10}=T^{death}\)) czyli:

\begin{itemize}
\tightlist
\item
  zaburzeniu podlega 1 parameter (4 symulacje po jednej dla
  `indywidualnie' zaburzonych parametrów \(p_7\ldots p_{10}\)),
\item
  zaburzeniu podlegaja 2 parametry (6 symulacji dla roznych par
  zaburzonych parametrów)
\item
  zaburzeniu podlegaja 3 parametry (4 symulacje dla jednego parametru
  niezaburzonego)
\item
  zaburzeniu podlegaja 4 parametry (1 symulacja dla wszystkich
  zaburzonych )
\end{itemize}

Z tym ze tym razem poza zaburzanymi i ewoluowanymi parametyrami
wszystkie pozostale byly ``fixed''. Czyli jesli zaburzamy pojedynyczy
parametr \(p_7\) to wszystkie pozostale (\(p_8 \ldots p_{10}\) powinny
zostac ``zafikswoane'' na swoich wartosciach referencyjnych i nie byc
poddawane ewolucji/poszukiwaniom.)

Podobnie jesli zaburzaniu ma podleagc para (np \({p_8}\) i \(p_{10}\))
to pozostale dwa bedace przedmiotem zainteresowania powinny zostac
zafiksowane na swoich wartosciach referencyjnych.

\subsection{Problemy na jakie
natrafiono}\label{problemy-na-jakie-natrafiono}

Niestety po wprowadzeniu do definicji solwera ewolucyjnego modyfikacji
pozwalajacych prowadzic eksperymenty zgodnie z powyzszym opisem nadziano
sie na problemy z dzialaniem solwera. Wydaje sie ze problemy mialy
podobna nature jak poprzednio - przy jakiejs kombinacji parametrow
nastepowal zbyt duzy rozrost (mapy) zyl co powodowalo, albo bardzo
znaczne wydluzenie obliczen albo ``wywrocenie'' sie solwera z uwagi na
brak mozliwosci dalszego alokowania pamieci etc.

\texttt{Unable\ to\ run\ solver:\ java.lang.NumberFormatException:\ For\ input\ string:\ "==================================================================================="\ mpirun\ -np\ 1\ -host\ node13:1\ ../TumorGenetyka/tumor\ 2\ 80\ 9900\ 0.1\ 1000\ 0.5\ 9.454590847505983\ 2.9222689984674073\ 9.063008773943382\ 96.1524747998914\ 0.001\ 0.3\ 0.625\ 0.3205\ 0.0064\ 0.0064\ 0.0000555\ 0.01\ 0.0000555\ 0.01\ 0.4\ 0.5\ 0.05\ 0.3\ 0.01333\ 10\ 0.003\ 2\ 5\ 25\ 24\ 0.003\ 0.4\ Unable\ to\ run\ solver:\ java.lang.NumberFormatException:\ For\ input\ string:\ "==================================================================================="\ mpirun\ -np\ 1\ -host\ node13:1\ ../TumorGenetyka/tumor\ 2\ 80\ 9900\ 0.1\ 1000\ 0.5\ 9.430679318110801\ 2.9613064430274214\ 9.001731455509889\ 94.84556414555985\ 0.001\ 0.3\ 0.625\ 0.3205\ 0.0064\ 0.0064\ 0.0000555\ 0.01\ 0.0000555\ 0.01\ 0.4\ 0.5\ 0.05\ 0.3\ 0.01333\ 10\ 0.003\ 2\ 5\ 25\ 24\ 0.003\ 0.4\ Unable\ to\ run\ solver:\ java.lang.NumberFormatException:\ For\ input\ string:\ "==================================================================================="\ mpirun\ -np\ 1\ -host\ node13:1\ ../TumorGenetyka/tumor\ 2\ 80\ 9900\ 0.1\ 1000\ 0.5\ 9.312816143866291\ 2.8373159607698737\ 9.068744087595107\ 109.4082352110131\ 0.001\ 0.3\ 0.625\ 0.3205\ 0.0064\ 0.0064\ 0.0000555\ 0.01\ 0.0000555\ 0.01\ 0.4\ 0.5\ 0.05\ 0.3\ 0.01333\ 10\ 0.003\ 2\ 5\ 25\ 24\ 0.003\ 0.4\ Unable\ to\ run\ solver:\ java.lang.NumberFormatException:\ For\ input\ string:\ "==================================================================================="\ mpirun\ -np\ 1\ -host\ node13:1\ ../TumorGenetyka/tumor\ 2\ 80\ 9900\ 0.1\ 1000\ 0.5\ 9.45412400070996\ 2.9512270566632774\ 9.134113928012875\ 96.1524747998914\ 0.001\ 0.3\ 0.625\ 0.3205\ 0.0064\ 0.0064\ 0.0000555\ 0.01\ 0.0000555\ 0.01\ 0.4\ 0.5\ 0.05\ 0.3\ 0.01333\ 10\ 0.003\ 2\ 5\ 25\ 24\ 0.003\ 0.4\ Unable\ to\ run\ solver:\ java.lang.NumberFormatException:\ For\ input\ string:\ "==================================================================================="\ mpirun\ -np\ 1\ -host\ node13:1\ ../TumorGenetyka/tumor\ 2\ 80\ 9900\ 0.1\ 1000\ 0.5\ 9.506700280337986\ 2.6729585594183844\ 9.127711756586537\ 98.28336497095415\ 0.001\ 0.3\ 0.625\ 0.3205\ 0.0064\ 0.0064\ 0.0000555\ 0.01\ 0.0000555\ 0.01\ 0.4\ 0.5\ 0.05\ 0.3\ 0.01333\ 10\ 0.003\ 2\ 5\ 25\ 24\ 0.003\ 0.4\ Unable\ to\ run\ solver:\ java.lang.NumberFormatException:\ For\ input\ string:\ "==================================================================================="\ mpirun\ -np\ 1\ -host\ node13:1\ ../TumorGenetyka/tumor\ 2\ 80\ 9900\ 0.1\ 1000\ 0.5\ 9.428205108285592\ 2.6603929941110316\ 9.0677395656585\ 97.18615629884131\ 0.001\ 0.3\ 0.625\ 0.3205\ 0.0064\ 0.0064\ 0.0000555\ 0.01\ 0.0000555\ 0.01\ 0.4\ 0.5\ 0.05\ 0.3\ 0.01333\ 10\ 0.003\ 2\ 5\ 25\ 24\ 0.003\ 0.4\ Unable\ to\ run\ solver:\ java.lang.NumberFormatException:\ For\ input\ string:\ "==================================================================================="\ mpirun\ -np\ 1\ -host\ node13:1\ ../TumorGenetyka/tumor\ 2\ 80\ 9900\ 0.1\ 1000\ 0.5\ 9.477124985184965\ 2.7878387123218205\ 9.071758596198237\ 108.77922017134355\ 0.001\ 0.3\ 0.625\ 0.3205\ 0.0064\ 0.0064\ 0.0000555\ 0.01\ 0.0000555\ 0.01\ 0.4\ 0.5\ 0.05\ 0.3\ 0.01333\ 10\ 0.003\ 2\ 5\ 25\ 24\ 0.003\ 0.4\ Read\ from\ err:\ terminate\ Read\ from\ err:\ called\ Read\ from\ err:\ after\ Read\ from\ err:\ throwing\ Read\ from\ err:\ an\ Read\ from\ err:\ instance\ Read\ from\ err:\ of\ Read\ from\ err:\ \textquotesingle{}std::bad\_alloc\textquotesingle{}\ Read\ from\ err:\ what():\ Read\ from\ err:\ std::bad\_alloc\ Unable\ to\ run\ solver:\ java.lang.NumberFormatException:\ For\ input\ string:\ "==================================================================================="\ mpirun\ -np\ 1\ -host\ node13:1\ ../TumorGenetyka/tumor\ 2\ 80\ 9900\ 0.1\ 1000\ 0.5\ 9.45300605302644\ 2.991957323550271\ 9.06939941685454\ 96.1524747998914\ 0.001\ 0.3\ 0.625\ 0.3205\ 0.0064\ 0.0064\ 0.0000555\ 0.01\ 0.0000555\ 0.01\ 0.4\ 0.5\ 0.05\ 0.3\ 0.01333\ 10\ 0.003\ 2\ 5\ 25\ 24\ 0.003\ 0.4\ Unable\ to\ run\ solver:\ java.lang.NumberFormatException:\ For\ input\ string:\ "==================================================================================="\ mpirun\ -np\ 1\ -host\ node13:1\ ../TumorGenetyka/tumor\ 2\ 80\ 9900\ 0.1\ 1000\ 0.5\ 9.424280747231732\ 2.816132673178266\ 9.080270122482826\ 109.95693142895279\ 0.001\ 0.3\ 0.625\ 0.3205\ 0.0064\ 0.0064\ 0.0000555\ 0.01\ 0.0000555\ 0.01\ 0.4\ 0.5\ 0.05\ 0.3\ 0.01333\ 10\ 0.003\ 2\ 5\ 25\ 24\ 0.003\ 0.4\ Unable\ to\ run\ solver:\ java.lang.NumberFormatException:\ For\ input\ string:\ "==================================================================================="\ mpirun\ -np\ 1\ -host\ node13:1\ ../TumorGenetyka/tumor\ 2\ 80\ 9900\ 0.1\ 1000\ 0.5\ 9.454330657752786\ 2.6073509233841365\ 9.065938202205565\ 109.70008853564136\ 0.001\ 0.3\ 0.625\ 0.3205\ 0.0064\ 0.0064\ 0.0000555\ 0.01\ 0.0000555\ 0.01\ 0.4\ 0.5\ 0.05\ 0.3\ 0.01333\ 10\ 0.003\ 2\ 5\ 25\ 24\ 0.003\ 0.4\ Unable\ to\ run\ solver:\ java.lang.NumberFormatException:\ For\ input\ string:\ "==================================================================================="\ mpirun\ -np\ 1\ -host\ node13:1\ ../TumorGenetyka/tumor\ 2\ 80\ 9900\ 0.1\ 1000\ 0.5\ 9.729087704398712\ 2.9222689984674073\ 9.146183592522865\ 98.30290819602372\ 0.001\ 0.3\ 0.625\ 0.3205\ 0.0064\ 0.0064\ 0.0000555\ 0.01\ 0.0000555\ 0.01\ 0.4\ 0.5\ 0.05\ 0.3\ 0.01333\ 10\ 0.003\ 2\ 5\ 25\ 24\ 0.003\ 0.}

\subsection{Zmiany wprowadzone do
solwera}\label{zmiany-wprowadzone-do-solwera}

Aby uniknac tego typu problemow, wprowadzono w solwerze nastepujaca
zmiane

Dodano kod który uzależnia wzrost żył od aktualnego rozmiaru już
istniejącego układu żył. Model żył działa w ten sposób, że w określonych
warunkach jest pewne prawdopodobieństwo że coś nowego się utworzy,
zmiana polega na tym ze to prawdopodobieństwo jest mnożone przez liczbę
inhibit\_factor z przedziału {[}0, 1{]}, która jest równa 1 dopóki ilość
wierzchołków + żył jest mniejsza niż 10,000 (czyli wtedy
prawdopodobieństwo utworzenia nowych jest takie samo jak przed tą
zmianą), a jak jest ta ilość większa to odpowiednio inhibit\_factor jest
mniejszy, dla układu z milionem+ staje się zerem. To powinno dość
sztywno ograniczyć rozrost niezależnie od parametrów, jednocześnie nie
zmieniając wiele dla ``rozsądnych'' symulacji gdzie tych żył się nie
robią miliony.

\subsection{Aktualna postac funkcji
celu}\label{aktualna-postac-funkcji-celu}

Po powyzszych zmianach, poniewaz zmienia to model, przleiczono ponownie
wartosci uzyskiwane przez solwer dla referencyjnych wartosci parametrow.
Wartość ta aktualnie to 397321

A zatem funkcja fitnes to aktualnie:

\begin{equation}
abs(397321-tumor_{volume})
 \end{equation}

Wartość ta byla minimalizowana podczas ewolucji.

\newpage

\subsection{Zaburzanie 4 parametrów}\label{zaburzanie-4-parametruxf3w}

\subsubsection{Zmiennosc parametrow}\label{zmiennosc-parametrow}

\begin{longtable}[c]{@{}lccc@{}}
\toprule
\begin{minipage}[b]{0.13\columnwidth}\raggedright\strut
Parametr
\strut\end{minipage} &
\begin{minipage}[b]{0.26\columnwidth}\centering\strut
Zakres zmienności podczas wstępnej ewaluacji
\strut\end{minipage} &
\begin{minipage}[b]{0.24\columnwidth}\centering\strut
Zakres zmienności dla populacji początkowej
\strut\end{minipage} &
\begin{minipage}[b]{0.26\columnwidth}\centering\strut
Zakres zmienności w trakcie trwania ewolucji
\strut\end{minipage}\tabularnewline
\midrule
\endhead
\begin{minipage}[t]{0.13\columnwidth}\raggedright\strut
\(p_7\)
\strut\end{minipage} &
\begin{minipage}[t]{0.26\columnwidth}\centering\strut
\(p_7 = 10 \pm 10\%\)
\strut\end{minipage} &
\begin{minipage}[t]{0.24\columnwidth}\centering\strut
\(p_7 \in [11.0,15.0]\)
\strut\end{minipage} &
\begin{minipage}[t]{0.26\columnwidth}\centering\strut
\(p_7 \in [5.0,15.0]\)
\strut\end{minipage}\tabularnewline
\begin{minipage}[t]{0.13\columnwidth}\raggedright\strut
\(p_8\)
\strut\end{minipage} &
\begin{minipage}[t]{0.26\columnwidth}\centering\strut
\(p_8 = 2 \pm 10 \%\)
\strut\end{minipage} &
\begin{minipage}[t]{0.24\columnwidth}\centering\strut
\(p_{8} \in [2.1,3.0]\)
\strut\end{minipage} &
\begin{minipage}[t]{0.26\columnwidth}\centering\strut
\(p_8 \in [1.0,3.0]\)
\strut\end{minipage}\tabularnewline
\begin{minipage}[t]{0.13\columnwidth}\raggedright\strut
\(p_9\)
\strut\end{minipage} &
\begin{minipage}[t]{0.26\columnwidth}\centering\strut
\(p_9 = 10 \pm 10 \%\)
\strut\end{minipage} &
\begin{minipage}[t]{0.24\columnwidth}\centering\strut
\(p_9 \in [11.0,15.0]\)
\strut\end{minipage} &
\begin{minipage}[t]{0.26\columnwidth}\centering\strut
\(p_9 \in [5.0,15.0]\)
\strut\end{minipage}\tabularnewline
\begin{minipage}[t]{0.13\columnwidth}\raggedright\strut
\(p_{10}\)
\strut\end{minipage} &
\begin{minipage}[t]{0.26\columnwidth}\centering\strut
\(p_{10} = 100 \pm 10 \%\)
\strut\end{minipage} &
\begin{minipage}[t]{0.24\columnwidth}\centering\strut
\(p_{10} \in [110.0,150.0]\)
\strut\end{minipage} &
\begin{minipage}[t]{0.26\columnwidth}\centering\strut
\(p_{10} \in [50.0,150.0]\)
\strut\end{minipage}\tabularnewline
\bottomrule
\end{longtable}

\subsubsection{Rozklad P7}\label{rozklad-p7}

\begin{figure}[htbp]
\centering
\includegraphics[width=0.75000\textwidth]{figures/popp7All4.png}
\caption{Rozklad \(p_7\) w populacji przy zaburzaniu 4 parametrow}
\end{figure}

\begin{figure}[htbp]
\centering
\includegraphics[width=0.75000\textwidth]{figures/agrP7All4.png}
\caption{Wartosci srednie, maksymalne oraz minimalne p7 na kolejnych
etapach dzialania alg ewolucyjnego przy zaburzaniu 4 parametrow}
\end{figure}

\begin{figure}[htbp]
\centering
\includegraphics[width=0.90000\textwidth]{figures/Odchylkap7All4.png}
\caption{Odchylka sredniej wartosci p7 w populacji od wartosci
referencyjnej przy zaburzaniu 4 parametrow}
\end{figure}

\begin{figure}[htbp]
\centering
\includegraphics[width=0.90000\textwidth]{figures/ProcOdchylkap7All4.png}
\caption{Procentowa Odchylka sredniej wartosci p7 w populacji od
wartosci referencyjnej przy zaburzaniu 4 parametrow}
\end{figure}

\newpage

\subsubsection{Rozklad P8}\label{rozklad-p8}

\begin{figure}[htbp]
\centering
\includegraphics[width=0.75000\textwidth]{figures/popp8All4.png}
\caption{Rozklad \(p_8\) w populacji przy zaburzaniu 4 parametrow}
\end{figure}

\begin{figure}[htbp]
\centering
\includegraphics[width=0.75000\textwidth]{figures/agrP8All4.png}
\caption{Wartosci srednie, maksymalne oraz minimalne p8 na kolejnych
etapach dzialania alg ewolucyjnego przy zaburzaniu 4 parametrow}
\end{figure}

\begin{figure}[htbp]
\centering
\includegraphics[width=0.75000\textwidth]{figures/Odchylkap8All4.png}
\caption{Odchylka sredniej wartosci p8 w populacji od wartosci
referencyjnej przy zaburzaniu 4 parametrow}
\end{figure}

\begin{figure}[htbp]
\centering
\includegraphics[width=0.75000\textwidth]{figures/ProcOdchylkap8All4.png}
\caption{Procentowa Odchylka sredniej wartosci p8 w populacji od
wartosci referencyjnej przy zaburzaniu 4 parametrow}
\end{figure}

\newpage

\subsubsection{Rozklad P9}\label{rozklad-p9}

\begin{figure}[htbp]
\centering
\includegraphics[width=0.75000\textwidth]{figures/popp9All4.png}
\caption{Rozklad \(p_9\) w populacji przy zaburzaniu 4 parametrow}
\end{figure}

\begin{figure}[htbp]
\centering
\includegraphics[width=0.75000\textwidth]{figures/agrP9All4.png}
\caption{Wartosci srednie, maksymalne oraz minimalne p9 na kolejnych
etapach dzialania alg ewolucyjnego przy zaburzaniu 4 parametrow}
\end{figure}

\begin{figure}[htbp]
\centering
\includegraphics[width=0.75000\textwidth]{figures/Odchylkap9All4.png}
\caption{Odchylka sredniej wartosci p9 w populacji od wartosci
referencyjnej przy zaburzaniu 4 parametrow}
\end{figure}

\begin{figure}[htbp]
\centering
\includegraphics[width=0.75000\textwidth]{figures/ProcOdchylkap9All4.png}
\caption{Procentowa Odchylka sredniej wartosci p9 w populacji od
wartosci referencyjnej przy zaburzaniu 4 parametrow}
\end{figure}

\newpage

\subsubsection{Rozklad P10}\label{rozklad-p10}

\begin{figure}[htbp]
\centering
\includegraphics[width=0.75000\textwidth]{figures/popp10All4.png}
\caption{Rozklad \(p_{10}\) w populacji przy zaburzaniu 4 parametrow}
\end{figure}

\begin{figure}[htbp]
\centering
\includegraphics[width=0.75000\textwidth]{figures/agrP10All4.png}
\caption{Wartosci srednie, maksymalne oraz minimalne p10 na kolejnych
etapach dzialania alg ewolucyjnego przy zaburzaniu 4 parametrow}
\end{figure}

\begin{figure}[htbp]
\centering
\includegraphics[width=0.75000\textwidth]{figures/Odchylkap10All4.png}
\caption{Odchylka sredniej wartosci p10 w populacji od wartosci
referencyjnej przy zaburzaniu 4 parametrow}
\end{figure}

\begin{figure}[htbp]
\centering
\includegraphics[width=0.75000\textwidth]{figures/ProcOdchylkap10All4.png}
\caption{Procentowa Odchylka sredniej wartosci p10 w populacji od
wartosci referencyjnej przy zaburzaniu 4 parametrow}
\end{figure}

\newpage

\subsubsection{Rozklad TumorVolume}\label{rozklad-tumorvolume}

\begin{figure}[htbp]
\centering
\includegraphics[width=0.75000\textwidth]{figures/popo1All4.png}
\caption{Rozklad \(TumorVolume\) w populacji przy zaburzaniu 4
parametrow}
\end{figure}

\begin{figure}[htbp]
\centering
\includegraphics[width=0.75000\textwidth]{figures/agro1All4.png}
\caption{Wartosci srednie, maksymalne oraz minimalne TumorVolume na
kolejnych etapach dzialania alg ewolucyjnego przy zaburzaniu 4
parametrow}
\end{figure}

\begin{figure}[htbp]
\centering
\includegraphics[width=0.75000\textwidth]{figures/Odchylkao1All4.png}
\caption{Odchylka sredniej wartosci TumorVolume w populacji od wartosci
referencyjnej przy zaburzaniu 4 parametrow}
\end{figure}

\begin{figure}[htbp]
\centering
\includegraphics[width=0.75000\textwidth]{figures/ProcOdchylkao1All4.png}
\caption{Procentowa Odchylka sredniej wartosci TumorVolume w populacji
od wartosci referencyjnej przy zaburzaniu 4 parametrow}
\end{figure}

\newpage

\subsubsection{Wnioski}\label{wnioski}

Ponizej odchylki wszyskiech 4 parametrow i tumorVolume od ich wartosci
referencyjncych.

\begin{figure}[htbp]
\centering
\includegraphics[width=0.75000\textwidth]{figures/ProcOdchylkaAll.png}
\caption{Procentowa Odchylka poszczegolnych parametrow i TumorVolume od
wart referencyjnych}
\end{figure}

\begin{figure}[htbp]
\centering
\includegraphics[width=0.75000\textwidth]{figures/1.png}
\caption{Zwiazek pomiedzy wartosciami parametrow \(p_7\), \(p_8\),
\(p_9\) i \(p_{10}\) a objetoscia guza}
\end{figure}

\newpage

Zbieznosc niby jest, ale nie wszedzie i niekoniecznie do wartosci
referencyjnych

O ile odchylka rozmiaru guza od wartosci referencyjnej zmierza do 0 to
juz od wartosciach poszczegolnych parametrow powiedziec tego nie mozna.

Co wiecej w przypadku p8 ta odchylka wrecz rosnie :) ale - brak
zbieznosci w przypadku p8 jest w zasadzie oczekiwany bo jak sie popatrzy
na wykres zaleznosci tumorvolume z wartoscia tego parametru to nie widac
tam trendu. Wraz ze zmianami p8 wartosci tumorvolume owszem zmienia sie
ale bez jakiegos specjalnego (a w zasadzie zadnego) trendu - wiec i
ewolucji trudno bylo podazac pewnie z p8 w jakims konkretnym kierunku.

\begin{itemize}
\item
  Troche zastanawiajace jest dlaczego efekt jest nieco inny w przypadku
  p10 skoro zaleznosc pomiedzy tumorVolume a tym parametrem byla podobna
  do zaleznosci od p8
\item
  to be continued\ldots{}..
\end{itemize}

\newpage

\newpage

\subsection{Zaburzanie 4 parametrów - wydluzone
dzialanie}\label{zaburzanie-4-parametruxf3w---wydluzone-dzialanie}

\subsubsection{Zmiennosc parametrow}\label{zmiennosc-parametrow-1}

\begin{longtable}[c]{@{}lccc@{}}
\toprule
\begin{minipage}[b]{0.13\columnwidth}\raggedright\strut
Parametr
\strut\end{minipage} &
\begin{minipage}[b]{0.26\columnwidth}\centering\strut
Zakres zmienności podczas wstępnej ewaluacji
\strut\end{minipage} &
\begin{minipage}[b]{0.24\columnwidth}\centering\strut
Zakres zmienności dla populacji początkowej
\strut\end{minipage} &
\begin{minipage}[b]{0.26\columnwidth}\centering\strut
Zakres zmienności w trakcie trwania ewolucji
\strut\end{minipage}\tabularnewline
\midrule
\endhead
\begin{minipage}[t]{0.13\columnwidth}\raggedright\strut
\(p_7\)
\strut\end{minipage} &
\begin{minipage}[t]{0.26\columnwidth}\centering\strut
\(p_7 = 10 \pm 10\%\)
\strut\end{minipage} &
\begin{minipage}[t]{0.24\columnwidth}\centering\strut
\(p_7 \in [11.0,15.0]\)
\strut\end{minipage} &
\begin{minipage}[t]{0.26\columnwidth}\centering\strut
\(p_7 \in [5.0,15.0]\)
\strut\end{minipage}\tabularnewline
\begin{minipage}[t]{0.13\columnwidth}\raggedright\strut
\(p_8\)
\strut\end{minipage} &
\begin{minipage}[t]{0.26\columnwidth}\centering\strut
\(p_8 = 2 \pm 10 \%\)
\strut\end{minipage} &
\begin{minipage}[t]{0.24\columnwidth}\centering\strut
\(p_{8} \in [2.1,3.0]\)
\strut\end{minipage} &
\begin{minipage}[t]{0.26\columnwidth}\centering\strut
\(p_8 \in [1.0,3.0]\)
\strut\end{minipage}\tabularnewline
\begin{minipage}[t]{0.13\columnwidth}\raggedright\strut
\(p_9\)
\strut\end{minipage} &
\begin{minipage}[t]{0.26\columnwidth}\centering\strut
\(p_9 = 10 \pm 10 \%\)
\strut\end{minipage} &
\begin{minipage}[t]{0.24\columnwidth}\centering\strut
\(p_9 \in [11.0,15.0]\)
\strut\end{minipage} &
\begin{minipage}[t]{0.26\columnwidth}\centering\strut
\(p_9 \in [5.0,15.0]\)
\strut\end{minipage}\tabularnewline
\begin{minipage}[t]{0.13\columnwidth}\raggedright\strut
\(p_{10}\)
\strut\end{minipage} &
\begin{minipage}[t]{0.26\columnwidth}\centering\strut
\(p_{10} = 100 \pm 10 \%\)
\strut\end{minipage} &
\begin{minipage}[t]{0.24\columnwidth}\centering\strut
\(p_{10} \in [110.0,150.0]\)
\strut\end{minipage} &
\begin{minipage}[t]{0.26\columnwidth}\centering\strut
\(p_{10} \in [50.0,150.0]\)
\strut\end{minipage}\tabularnewline
\bottomrule
\end{longtable}

Zmiana w tym eksperymencie w stosunku do poprzednio opisywanego polegala
na wydluzeniu dzialania alg ewolucyjnego do 600 ewaluacji (600 wywolan
solwera)

\subsubsection{Rozklad P7}\label{rozklad-p7-1}

\begin{figure}[htbp]
\centering
\includegraphics[width=0.75000\textwidth]{figures/popp7All4_600Ev.png}
\caption{Rozklad \(p_7\) w populacji przy zaburzaniu 4 parametrow dla
600 ewaluacji}
\end{figure}

\begin{figure}[htbp]
\centering
\includegraphics[width=0.75000\textwidth]{figures/agrP7All4_600Ev.png}
\caption{Wartosci srednie, maksymalne oraz minimalne p7 na kolejnych
etapach dzialania alg ewolucyjnego przy zaburzaniu 4 parametrow dla 600
ewaluacji}
\end{figure}

\begin{figure}[htbp]
\centering
\includegraphics[width=0.90000\textwidth]{figures/Odchylkap7All4_600Ev.png}
\caption{Odchylka sredniej wartosci p7 w populacji od wartosci
referencyjnej przy zaburzaniu 4 parametrow dla 600 ewaluacji}
\end{figure}

\begin{figure}[htbp]
\centering
\includegraphics[width=0.90000\textwidth]{figures/ProcOdchylkap7All4_600Ev.png}
\caption{Procentowa Odchylka sredniej wartosci p7 w populacji od
wartosci referencyjnej przy zaburzaniu 4 parametrow}
\end{figure}

\newpage

\subsubsection{Rozklad P8}\label{rozklad-p8-1}

\begin{figure}[htbp]
\centering
\includegraphics[width=0.75000\textwidth]{figures/popp8All4_600Ev.png}
\caption{Rozklad \(p_8\) w populacji przy zaburzaniu 4 parametrow dla
600 ewaluacji}
\end{figure}

\begin{figure}[htbp]
\centering
\includegraphics[width=0.75000\textwidth]{figures/agrP8All4_600Ev.png}
\caption{Wartosci srednie, maksymalne oraz minimalne p8 na kolejnych
etapach dzialania alg ewolucyjnego przy zaburzaniu 4 parametrow dla 600
ewaluacji}
\end{figure}

\begin{figure}[htbp]
\centering
\includegraphics[width=0.75000\textwidth]{figures/Odchylkap8All4_600Ev.png}
\caption{Odchylka sredniej wartosci p8 w populacji od wartosci
referencyjnej przy zaburzaniu 4 parametrow dla 600 ewaluacji}
\end{figure}

\begin{figure}[htbp]
\centering
\includegraphics[width=0.75000\textwidth]{figures/ProcOdchylkap8All4_600Ev.png}
\caption{Procentowa Odchylka sredniej wartosci p8 w populacji od
wartosci referencyjnej przy zaburzaniu 4 parametrow dla 600 ewaluacji}
\end{figure}

\newpage

\subsubsection{Rozklad P9}\label{rozklad-p9-1}

\begin{figure}[htbp]
\centering
\includegraphics[width=0.75000\textwidth]{figures/popp9All4_600Ev.png}
\caption{Rozklad \(p_9\) w populacji przy zaburzaniu 4 parametrow dla
600 ewaluacji}
\end{figure}

\begin{figure}[htbp]
\centering
\includegraphics[width=0.75000\textwidth]{figures/agrP9All4_600Ev.png}
\caption{Wartosci srednie, maksymalne oraz minimalne p9 na kolejnych
etapach dzialania alg ewolucyjnego przy zaburzaniu 4 parametrow dla 600
ewaluacji}
\end{figure}

\begin{figure}[htbp]
\centering
\includegraphics[width=0.75000\textwidth]{figures/Odchylkap9All4_600Ev.png}
\caption{Odchylka sredniej wartosci p9 w populacji od wartosci
referencyjnej przy zaburzaniu 4 parametrow dla 600 ewaluacji}
\end{figure}

\begin{figure}[htbp]
\centering
\includegraphics[width=0.75000\textwidth]{figures/ProcOdchylkap9All4_600Ev.png}
\caption{Procentowa Odchylka sredniej wartosci p9 w populacji od
wartosci referencyjnej przy zaburzaniu 4 parametrow dla 600 ewaluacji}
\end{figure}

\newpage

\subsubsection{Rozklad P10}\label{rozklad-p10-1}

\begin{figure}[htbp]
\centering
\includegraphics[width=0.75000\textwidth]{figures/popp10All4_600Ev.png}
\caption{Rozklad \(p_{10}\) w populacji przy zaburzaniu 4 parametrow dla
600 ewaluacji}
\end{figure}

\begin{figure}[htbp]
\centering
\includegraphics[width=0.75000\textwidth]{figures/agrP10All4_600Ev.png}
\caption{Wartosci srednie, maksymalne oraz minimalne p10 na kolejnych
etapach dzialania alg ewolucyjnego przy zaburzaniu 4 parametrow dla 600
ewaluacji}
\end{figure}

\begin{figure}[htbp]
\centering
\includegraphics[width=0.75000\textwidth]{figures/Odchylkap10All4_600Ev.png}
\caption{Odchylka sredniej wartosci p10 w populacji od wartosci
referencyjnej przy zaburzaniu 4 parametrow dla 600 ewaluacji}
\end{figure}

\begin{figure}[htbp]
\centering
\includegraphics[width=0.75000\textwidth]{figures/ProcOdchylkap10All4_600Ev.png}
\caption{Procentowa Odchylka sredniej wartosci p10 w populacji od
wartosci referencyjnej przy zaburzaniu 4 parametrow dla 600 ewaluacji}
\end{figure}

\newpage

\subsubsection{Rozklad TumorVolume}\label{rozklad-tumorvolume-1}

\begin{figure}[htbp]
\centering
\includegraphics[width=0.75000\textwidth]{figures/popo1All4_600Ev.png}
\caption{Rozklad \(TumorVolume\) w populacji przy zaburzaniu 4
parametrow dla 600 ewaluacji}
\end{figure}

\begin{figure}[htbp]
\centering
\includegraphics[width=0.75000\textwidth]{figures/agro1All4_600Ev.png}
\caption{Wartosci srednie, maksymalne oraz minimalne TumorVolume na
kolejnych etapach dzialania alg ewolucyjnego przy zaburzaniu 4
parametrow dla 600 ewaluacji}
\end{figure}

\begin{figure}[htbp]
\centering
\includegraphics[width=0.75000\textwidth]{figures/Odchylkao1All4_600Ev.png}
\caption{Odchylka sredniej wartosci TumorVolume w populacji od wartosci
referencyjnej przy zaburzaniu 4 parametrow dla 600 ewaluacji}
\end{figure}

\begin{figure}[htbp]
\centering
\includegraphics[width=0.75000\textwidth]{figures/ProcOdchylkao1All4_600Ev.png}
\caption{Procentowa Odchylka sredniej wartosci TumorVolume w populacji
od wartosci referencyjnej przy zaburzaniu 4 parametrow dla 600
ewaluacji}
\end{figure}

\newpage

\subsubsection{Wnioski}\label{wnioski-1}

Ponizej odchylki wszyskiech 4 parametrow i tumorVolume od ich wartosci
referencyjncych.

\begin{figure}[htbp]
\centering
\includegraphics[width=0.75000\textwidth]{figures/ProcOdchylkaAll_600Ev.png}
\caption{Procentowa Odchylka poszczegolnych parametrow i TumorVolume od
wart referencyjnych dla 600 ewaluacji}
\end{figure}

\newpage

\subsection{Zaburzanie 1 parametru}\label{zaburzanie-1-parametru}

\subsubsection{Zaburzanie P7}\label{zaburzanie-p7}

\paragraph{Zmiennosc parametru}\label{zmiennosc-parametru}

\begin{longtable}[c]{@{}lccc@{}}
\toprule
\begin{minipage}[b]{0.13\columnwidth}\raggedright\strut
Parametr
\strut\end{minipage} &
\begin{minipage}[b]{0.26\columnwidth}\centering\strut
Zakres zmienności podczas wstępnej ewaluacji
\strut\end{minipage} &
\begin{minipage}[b]{0.24\columnwidth}\centering\strut
Zakres zmienności dla populacji początkowej
\strut\end{minipage} &
\begin{minipage}[b]{0.26\columnwidth}\centering\strut
Zakres zmienności w trakcie trwania ewolucji
\strut\end{minipage}\tabularnewline
\midrule
\endhead
\begin{minipage}[t]{0.13\columnwidth}\raggedright\strut
\(p_7\)
\strut\end{minipage} &
\begin{minipage}[t]{0.26\columnwidth}\centering\strut
\(p_7 = 10 \pm 10\%\)
\strut\end{minipage} &
\begin{minipage}[t]{0.24\columnwidth}\centering\strut
\(p_7 \in [11.0,15.0]\)
\strut\end{minipage} &
\begin{minipage}[t]{0.26\columnwidth}\centering\strut
\(p_7 \in [5.0,15.0]\)
\strut\end{minipage}\tabularnewline
\begin{minipage}[t]{0.13\columnwidth}\raggedright\strut
\(p_8\)
\strut\end{minipage} &
\begin{minipage}[t]{0.26\columnwidth}\centering\strut
\(p_8 = 2 \pm 10 \%\)
\strut\end{minipage} &
\begin{minipage}[t]{0.24\columnwidth}\centering\strut
\(p_8= 2\)
\strut\end{minipage} &
\begin{minipage}[t]{0.26\columnwidth}\centering\strut
\(p_8 =2\)
\strut\end{minipage}\tabularnewline
\begin{minipage}[t]{0.13\columnwidth}\raggedright\strut
\(p_9\)
\strut\end{minipage} &
\begin{minipage}[t]{0.26\columnwidth}\centering\strut
\(p_9 = 10 \pm 10 \%\)
\strut\end{minipage} &
\begin{minipage}[t]{0.24\columnwidth}\centering\strut
\(p_9 =10\)
\strut\end{minipage} &
\begin{minipage}[t]{0.26\columnwidth}\centering\strut
\(p_9 =10\)
\strut\end{minipage}\tabularnewline
\begin{minipage}[t]{0.13\columnwidth}\raggedright\strut
\(p_{10}\)
\strut\end{minipage} &
\begin{minipage}[t]{0.26\columnwidth}\centering\strut
\(p_{10} = 100 \pm 10 \%\)
\strut\end{minipage} &
\begin{minipage}[t]{0.24\columnwidth}\centering\strut
\(p_{10} =100\)
\strut\end{minipage} &
\begin{minipage}[t]{0.26\columnwidth}\centering\strut
\(p_{10} =100\)
\strut\end{minipage}\tabularnewline
\bottomrule
\end{longtable}

\paragraph{Rozklad p7}\label{rozklad-p7-2}

\begin{figure}[htbp]
\centering
\includegraphics[width=0.75000\textwidth]{figures/popp7All1.png}
\caption{Rozklad \(p_7\) w populacji przy zaburzaniu 1 parametru (P7)}
\end{figure}

\begin{figure}[htbp]
\centering
\includegraphics[width=0.75000\textwidth]{figures/agrP7All1.png}
\caption{Wartosci srednie, maksymalne oraz minimalne p7 na kolejnych
etapach dzialania alg ewolucyjnego przy zaburzaniu p7}
\end{figure}

\begin{figure}[htbp]
\centering
\includegraphics[width=0.90000\textwidth]{figures/Odchylkap7All1.png}
\caption{Odchylka sredniej wartosci p7 w populacji od wartosci
referencyjnej przy zaburzaniu p7}
\end{figure}

\begin{figure}[htbp]
\centering
\includegraphics[width=0.90000\textwidth]{figures/ProcOdchylkap7All1.png}
\caption{Procentowa Odchylka sredniej wartosci p7 w populacji od
wartosci referencyjnej przy zaburzaniu p7}
\end{figure}

\newpage

\paragraph{Rozklad TumorVolume}\label{rozklad-tumorvolume-2}

\begin{figure}[htbp]
\centering
\includegraphics[width=0.75000\textwidth]{figures/popo1All1.png}
\caption{Rozklad \(TumorVolume\) w populacji przy zaburzaniu p7}
\end{figure}

\begin{figure}[htbp]
\centering
\includegraphics[width=0.75000\textwidth]{figures/agro1All1.png}
\caption{Wartosci srednie, maksymalne oraz minimalne TumorVolume na
kolejnych etapach dzialania alg ewolucyjnego przy zaburzaniu p7}
\end{figure}

\begin{figure}[htbp]
\centering
\includegraphics[width=0.75000\textwidth]{figures/Odchylkao1All1.png}
\caption{Odchylka sredniej wartosci TumorVolume w populacji od wartosci
referencyjnej przy zaburzaniu p7}
\end{figure}

\begin{figure}[htbp]
\centering
\includegraphics[width=0.75000\textwidth]{figures/ProcOdchylkao1All1.png}
\caption{Procentowa Odchylka sredniej wartosci TumorVolume w populacji
od wartosci referencyjnej przy zaburzaniu p7}
\end{figure}

\end{document}
